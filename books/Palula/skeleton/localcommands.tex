%add all your local new commands to this file

\newcommand{\smiley}{:)}

%Spitzmarken als spitzmarke
\def\spitzmarke{\paragraph*}

%Kein Abstand zwischen § und Abschnittsnummer
\renewcommand{\sectref}[1]{{\S}\ref{#1}}

%neuer Counter für Photofigure
\newcounter{photofigure}

%neue Umgebung für Photofigure
\newenvironment{photofigure}[1][]{\addtocounter{figure}{-1}\refstepcounter{photofigure}\renewcommand{\thefigure}{\thechapter.\arabic{photofigure}}\renewcommand{\figurename}{Photo}\begin{figure}[#1]}{\end{figure}}

%neuer Counter für Mapfigure
\newcounter{mapfigure}

%neue Umgebung für Mapfigure
\newenvironment{mapfigure}[1][]{\addtocounter{figure}{-1}\refstepcounter{mapfigure}\renewcommand{\thefigure}{\thechapter.\arabic{mapfigure}}\renewcommand{\figurename}{Map}\begin{figure}[#1]}{\end{figure}}

%am Kapitelanfang resetten:
\makeatletter
\@addtoreset{photofigure}{chapter}
\@addtoreset{mapfigure}{chapter}
\makeatother

%neuer Spaltentyp
\newcolumntype{Q}{>{\raggedright\arraybackslash}X}
\newcolumntype{C}{>{\centering\arraybackslash}X}
\newcolumntype{P}{>{\raggedright\hspace{0pt}\arraybackslash}p}

%Formatierung und Position der Nummer bei Tabellen in exe-Umgebung korrigieren
\def\extab{\ex\leavevmode\vadjust{\vspace{-\baselineskip}}\newline\normalfont}

